\msection{Broader Impact}

This proposal reflects the first, to our knowledge, structured
partnership between neuroscience and data science at Yale, and likewise the first between Brown's Data Science Initiative and its neuroscience community as such. The partnership
presents myriad new opportunities for promoting teaching, training, and
learning: First, the project will support two graduate students, one in
Psychology and the other in Statistics \& Data Science (S\&DS). They will
work in close coordination with senior personnel from both neuroscience
and data science, to gain mastery of both fields and to serve as a vital intellectual bridge. Second, several other
graduate students at Yale and Brown will participate in research
projects, workshops, and annual exchanges between institutions to learn
from other researchers and the communities. Third, neuroscience and data
science are the two newest undergraduate majors in Yale College, and the
assembled team is deeply invested in these programs (e.g., Turk-Browne
and Clark serve as co-directors of the neuroscience major). This project
will facilitate interaction between the majors, including a data science
course with neuroscience lab (Lafferty) and a neuroscience course
teaching data science approaches (Turk-Browne). These courses will be
cross-listed and coordinated so interested students can take both.
Fourth, seniors in our departments complete a year of independent
research. The project objectives provide opportunities for at least four
undergraduates per year to gain cutting-edge research experience. Joint
training in our fields is invaluable preparation for the STEM workforce,
where computational and mathematical skills paired with domain knowledge
in behavior, neuroscience, and artificial intelligence are highly sought
after. Fifth, the two workshops we organize, at Yale in year 1 on
neuroscience and at Brown in year 2 on data science, will be geared at
an accessible level and open to students and faculty on campus, to build
a broader community.

This collaboration involves an unusually broad constellation of people,
expertise, methods, and resources. Thus, the project activities will
enhance infrastructure for research and education in several ways:
First, the project will bring together four departments at Yale that
have not traditionally had substantive interactions: Psychology
(Turk-Browne), Molecular, Cellular, \& Developmental Biology (Clark),
Statistics and Data Science (Lafferty), Computer Science (Lafferty), and Mathematics (Brock). 
Second, it will establish a
new bridge between Yale and Brown, institutions that are relatively
nearby (short train ride away) and that have made significant
investments in neuroscience and data science. Combining forces, in terms
of faculty interaction and student exchanges, will multiply the impact
of these initiatives, and we hope lead to a regional network focused on
data neuroscience. Third, there are many relevant efforts underway on
each campus with which we can partner to further realize value from this
project. At Yale, these include: the Yale Center for Research Computing,
which manages centralized clusters and offers boot camps in scientific
programming, parallel computing, and software development; the human
brain imaging center in the Faculty of Arts and Science (overseen by
Turk-Browne) that will open this fall next to Psychology and S\&DS,
providing training for FAS students and faculty on how to conduct fMRI
studies and generous funding to enable studies without external grant
support; and the Quantitative Biology Institute (Q-Bio), an hub of
computational scientists (including Clark) that will provide additional
research opportunities. At Brown, relevant initiatives include: the NSF
EPSCoR Attention Consortium (Sheinberg is Co-PI), a multi-institution
effort to develop a unified model of attention, relevant to some of the
proposed research; the NSF-funded Institute for Computational and
Experimental Research in Mathematics (ICERM), which bridged mathematics
and computation (with founding co-PI's Brock and Sandstede serving as inaugural Deputy and Associate Director respectively);
and the Data Science Initiative supported by TRIPODS (Brock is
Director, Sandstede this summer) to be co-located in new space with Brown's Carney Institute for Brain Science in January 2019. Fourth, the activities supported by
this project have relevance to the technology industry, creating
opportunities for new partnerships and dissemination. As an example,
Turk-Browne has been a core member of a partnership with Intel Labs
started in 2015 that funds academic research and researchers in advanced
fMRI analysis, but also led to the creation of an internal group at
Intel Labs devoted to brain-inspired computing and intelligence.

