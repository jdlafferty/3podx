\def\person#1{\vskip4pt\noindent{\it\bfseries #1}}
\let\blurb\person

\section*{Project Coordination Plan}

\person{Roles and responsibilities}: This proposal includes eight key
personnel, evenly divided between data science and neuroscience, and
between Yale and Brown, the latter being the home of a TRIPODS
institute. The TRIPODS+X solicitation initiated conversations that led
to this proposal. In preparing it, we discovered previously unrealized
connections and shared interests within Yale and between Yale and
Brown. As such, funding this proposal will provide resources needed to
launch a new interdisciplinary collaboration at the intersection of
data science and neuroscience that would otherwise not be possible.

\person{Nick Turk-Browne} is a cognitive neuroscientist at Yale and will serve
as the lead PI. He will be responsible for administrative oversight of
the project, including being the contact for NSF, reviewing the budget
and expenditures quarterly, and preparing reports. His technical
responsibilities will be to procure existing fMRI datasets, format and
preprocess these data, and develop software packages that implement
current and new fMRI analysis tools (e.g., \href{http://www.brainiak.org}{brainiak.org}). His
scientific responsibilities will include mentoring a graduate student
in neuroscience and generating hypotheses and models of
representation, attention, and memory (objectives 2-4, respectively)
and interpreting findings based on theories of human brain function.


\person{John Lafferty} is a data scientist at Yale and will serve as Co-PI. He
will be responsible for leading the data science aspects of the
project (in all objectives). This includes lending his expertise in
machine learning, identifying and creating relevant algorithms,
developing precise mathematical formulations, and considering how
principles of the brain could be imported to enhance approaches in
data science. He will mentor a graduate student in data science and
teach a neuroscience lab as part of a new data science course.

\person{Damon Clark} is a cellular and computational neuroscientist at Yale and
will serve as Co-PI. He will be responsible for providing and
analyzing neuronal recordings from {\it Drosophila} and for developing
computational models of circuit function (objective 1). He will
co-supervise the graduate student in neuroscience so that s/he can
learn how computational principles apply across different species'
brains.

\person{Jeff Brock} is a mathematician and currently PI for the
TRIPODS grant at Brown. He was recently appointed as Professor of Mathematics and Dean
of Science at Yale, and will soon transition to Co-PI for the TRIPODS
grant at Brown. He will serve as Co-PI on this project, responsible for
close coordination with the aims of TRIPODS, identifying mathematical
concepts relevant to the objectives of this project (e.g., topological
methods), as well as communicating project findings to the full
TRIPODS team. He will co-supervise the graduate student in data
science to ensure that s/he learns about other application domains.

\person{Marvin Chun} is a cognitive neuroscientist at Yale and will serve as
Faculty Associate. He will be responsible for lending expertise in
human attention and memory (objectives 3-4, respectively) and
particularly the use of reconstruction models in fMRI analysis.

\person{Bjorn Sandstede} is an applied mathematician at Brown and will serve as
Faculty Associate. He will become PI of the TRIPODS grant at Brown. He will
be the primary contact at Brown, responsible for coordinating that
team of faculty and students. He has also worked on fMRI connectivity
and causal inference from networks and so will play an active
scientific role in the work on attentional filtering (objective 3).

\person{Stu Geman} is an applied mathematician at Brown and will serve as
Faculty Associate. He will be responsible for lending mathematical
expertise in neural coding (central to objectives 2 and 4).

\person{David Sheinberg} is a systems neuroscientist at Brown and will serve as
Faculty Associate. As an expert in non-human primates, who exhibit
similar behaviors to humans but allow for invasive recordings as in
{\it Drosophila}, he will serve to bridge across objectives and
techniques. His scientific focus on perception, attention, and memory
will benefit many of the proposed research activities.

\person{Expertise in neuroscience}: The ``X'' application in this application is
represented extremely well by the team of four neuroscientists above
(Turk-Browne, Clark, Chun, Sheinberg). We cover the spectrum of model
systems (humans, monkeys, flies). We employ most modern
neuroscientific techniques (neuronal recordings, optogenetics and
stimulation, cellular and whole-brain imaging, brain-damaged patients,
neural network modeling). We have continuous funding for neuroscience
research from NIH and/or NSF. We serve on editorial boards, advisory
boards, and grant panels for neuroscience research. Our labs give
multiple conference presentations every year at neuroscience
conferences like SfN. We have appointments, teach, and mentor students
in neuroscience graduate programs. Two of us (Turk-Browne and Clark)
are the co-directors of the neuroscience major at Yale. To go into
more detail about the PI (Turk-Browne), he has published 35
peer-reviewed articles containing one or more fMRI studies. He has
received major funding from NIH (2 R01s), NSF, Templeton Foundation,
and Intel Labs in support of his fMRI research. He has received 6
early career contribution awards, including from the Cognitive
Neuroscience Society. He has pioneered new fMRI approaches, including
background connectivity, closed-loop neurofeedback, full correlation
matrix analysis, hippocampal segmentation, and imaging of awake and
behaving infants. He has trained 7 PhD students and 6 postdocs in
cognitive neuroscience (10 of whom hold faculty positions, including
at: UBC, UPenn, UCSD, Oregon, Columbia, UCL). He has a track record of
interdisciplinary collaboration, including with computational
researchers in systems, networking, and machine learning. 

\blurb{Managing the collaboration}: The PI, Co-PIs, and Sandstede will
comprise an executive committee that will meet monthly. Turk-Browne
will be responsible for managing cognitive neuroscience activities,
Lafferty for data science activities, Clark for cellular neuroscience
activities, Brock for coordinating with TRIPODS, and Sandstede for
leading the Brown team. This committee will review project progress,
identify areas for improvement, make strategic personnel and research
decisions, resolve conflicts, and work toward long-term funding. The
regular interaction and proximity of Yale researchers (all within 2
blocks) and the close ties between Brock at Yale and Brown personnel
will ensure responsible execution of this project.

\blurb{Coordination mechanisms}: We envision four mechanisms to learn from
each other and to make progress on project objectives: (1) a weekly
Zoom videoconference with faculty and students, involving a business
meeting followed by a scientific presentation; (2) annual exchanges
for a few days of 4 Brown students to Yale and 4 Yale students to
Brown; (3) a workshop at Yale in year 1 focused on neuroscience; (4) a
workshop at Brown in year 2 focused on data science. Each workshop
will begin with activities that introduce the kinds of data and
methods unique to the project and discipline, at a level suitable for
graduate students. This will be followed by a series of lectures by
more senior personnel explaining the active work and progress along
the various objectives. The workshops will be designed to invite
others on campus to build community. Funding for the exchanges and
workshops is included in the proposed budget.

\vskip10pt
\blurb{Timeline of activities and milestones}

\def\graycell{\cellcolor[RGB]{200,200,200}}
\def\acell{\cellcolor[rgb]{.98,.81,.69}}
\def\bcell{\cellcolor[rgb]{0.54, 0.81, 0.94}}
\def\ccell{\cellcolor[rgb]{0.98, 0.91, 0.71}}
\def\dcell{\cellcolor[rgb]{0.64, 0.76, 0.68}}
\def\ecell{\cellcolor[rgb]{0.96, 0.76, 0.76}}
\def\g{}
\def\g{\graycell}
\def\a{\acell}
\def\b{\bcell}
\def\c{\ccell}
\def\d{\dcell}
\def\e{\ecell}
\def\yale{\bcell}
\def\brown{\acell}
\def\both{\bcell}


\def\topic#1{\multicolumn{13}{c}{}\\
  \multicolumn{1}{l}{\bf #1 } & \multicolumn{12}{c}{} \\[3pt] \hline}
\def\numb#1{\hbox to 10pt{\hfill #1\hfill}}
\def\ffour{\numb{4}}
\def\three{\numb{3}}
\def\two{\numb{2}}
\def\one{\numb{1}}
\def\four{\multicolumn{1}{c|}{\ffour}}
\def\col{C}
\def\stan{S}
\def\rice{R}

%\begin{figure*}
\noindent
\begin{small}
\begin{center}
\mbox{\ }\vskip20pt
\setlength{\tabcolsep}{3pt}    
\renewcommand{\arraystretch}{1.2}
\begin{tabular}{|>{\arraybackslash}m{2.5in}|c|c|c|c||c|c|c|c||c|c|c|c|}
\hline 
\multirow{2}{*}{\small \bf Activity} & \multicolumn{4}{c}{\bf
  Year 1} \vline & \multicolumn{4}{c}{\bf Year 2} \vline &
\multicolumn{4}{c|}{\bf Year 3} \\[3pt] \cline{2-13}
& \one & \two & \three & \four & \one & \two & \three & \four & \one & \two & \three & \four \\ \hline

%\topic{Research projects}
%Distributed processing & \a & \a & \a & \a & \a & \a & \a & \a & \a & \a & \a & \a  \\ \hline
%Data representation  & \a & \a & \a & \a & \a & \a & \a & \a & \a & \a & \a & \a  \\ \hline
%Attentional filtering & & & & & \a & \a & \a  & \a & \a & \a & &  \\ \hline
%Memory capacity & & & & &  & & & \a & \a & \a & \a & \a \\ \hline

%\topic{Project coordination}
\multicolumn{13}{c}{}\\[-12pt]
\hline

Coordination with TRIPODS & & \brown & & & & \brown & & & & \brown & & \\ \hline
Executive committee meetings & \b & \b & \b & \b & \b & \b & \b & \b & \b & \b & \b & \b \\ \hline
Student exchange & & & \brown & & & & \yale & & & & \yale & \\ \hline
Workshop &  &  &  & \yale &  &  &  & \brown & & & & \brown \\ \hline
Undergraduate data neuroscience lab &  &  \yale &  & &  & \yale &  & & & \yale & & \\ \hline
\end{tabular}
\end{center}
\end{small}
%\end{figure*}



