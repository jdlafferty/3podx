
\begin{center}
\ 
\vskip-30pt
{{\Large\bf  TRIPODS+X:RES: Investigations at the Interface of \\[5pt]
Data Science and Neuroscience
}}\\[10pt]
{\it Project Summary}
\vskip20pt
\end{center}


\newcounter{aim}
\def\aim#1{\vskip5pt\noindent {\stepcounter{aim}\it\bfseries Objective
    \arabic{aim}:\;\;#1.}\enspace}
\let\objective\aim

The research in this proposal explores the interface between
data science and neuroscience. The overarching theme is to develop
a two-way channel between these fields. 
In one direction, we will investigate how computational
principles from data science can be leveraged to understand recent empirical
findings and advance theory at different levels of neuroscience, 
including cellular measurements in fruit flies and 
whole-brain functional imaging in humans. In the
reverse direction, the project will view the processes and mechanisms
of vision and cognition underlying these findings as a source for
new statistical and mathematical frameworks for data analysis. The research
will focus on four interrelated objectives:


\objective{Distributed processing}
Recent work in machine learning has studied the effect of
communication constraints and parallelization in distributed
estimation. There is a close analogy to vision, where any given input
is sensed by multiple parts of the retina, from which an accurate percept
needs to be constructed. The project will consider different models
for distributed processing, motivated by visual learning and perception in
lower-level organisms. 


\objective{Data representation}
The brain stores the same information in several different ways, each
emphasizing different dimensions of the input. 
%In visual cortex, two
%inputs with similar visual features will be stored together. In
%temporal and frontal cortex, two inputs with the same conceptual or
%semantic meaning/function will be stored together. In the hippocampus, these kinds of
%sensory or semantic overlap are discounted by orthogonalizing similar
%inputs encountered at different times.
Inspired by current understanding of representation
of information in different regions of the cortex, the project will
investigate how parallel lossy representations of the same inputs
along different dimensions can be used for statistically and computationally efficient learning 
algorithms.  In the other direction, we will investigate how embedding
algorithms from machine learning might be used as mechanisms for
processing massive cellular and brain imaging data.

\objective{Attentional filtering} The project will develop
attention-based models in statistical learning, based on the use
of lower-dimensional traces or curves through a high-dimensional
input space. Attention curves have analogues in human cognition, where
input dimensions are processed based on their inherent salience and
relevance to a person's goals. The project will explore mathematical,
computational and empirical models of attention. Experiments will
focus on a public dataset of subjects watching episodes of Sherlock
while being scanned with fMRI, using a frame-by-frame annotation of
several dimensions of the movie as the basis for attention-based models.


\objective{Memory capacity} Evidence from studies of human behavior
suggests that people store information about objects and events in
long-term memory with incredible detail. How is this possible?  
We will consider cognitive studies and
a current understanding of possible memory architectures in natural systems
in order to inform approaches for reducing and sharing memory in
artificial learning algorithms. A framework will be developed for
establishing lower bounds on the risk of machine learning algorithms under
memory constraints. Insights from this mathematical theory will be
considered in the context of memory of complex organisms.



\vskip8pt

\noindent 
The {\it\bfseries intellectual merit} of the proposed research
includes the transfer of ideas between data science and neuroscience,
with the goal of advancing knowledge in both domains.
The {\it\bfseries broader impact} of the research includes development of software that
implements the advanced data science and machine learning algorithms,
the development of labs for an undergraduate course 
in data science with examples drawn from neuroscience, and a series
of workshops hosted at Yale and Brown Universities that expand
the scope of the original TRIPODS effort at Brown.

