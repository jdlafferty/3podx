
\begin{center}
\ 
\vskip-30pt
{{\Large\bf  TRIPODS+X:RES: Investigations at the Interface of \\[5pt]
Data Science and Neuroscience
}}\\[10pt]
{\it Project Summary}
\vskip20pt
\end{center}


\newcounter{aim}
\def\aim#1{\vskip5pt\noindent {\stepcounter{aim}\it\bfseries Objective
    \arabic{aim}:\;\;#1.}\enspace}
\let\objective\aim

The research in this proposal explores the interface between
data science and neuroscience. The overarching theme is to develop
a two-way channel between these fields. 
In one direction, the project will investigate how computational
principles from data science can be leveraged to understand recent empirical
findings and advance theory at different levels of neuroscience, 
including cellular measurements in fruit flies and 
whole-brain functional imaging in humans. In the
reverse direction, the project will view the processes and mechanisms
of vision and cognition underlying these findings as a source for
new statistical and mathematical frameworks for data analysis. The research
will focus on four interrelated objectives:


\objective{Distributed processing} Recent work in machine learning has studied the effect of
communication constraints and parallelization in distributed
estimation. There is a close analogy to biological vision, 
where any given visual input is sensed by multiple parts of the retina, 
from which an accurate percept needs to be constructed. The project will consider different models for distributed processing, motivated by findings concerning visual learning and motion perception in lower-level organisms. 


\objective{Data representation} The brain stores the same information in several different ways, each
emphasizing different dimensions of the input. 
%In visual cortex, two
%inputs with similar visual features will be stored together. In
%temporal and frontal cortex, two inputs with the same conceptual or
%semantic meaning/function will be stored together. In the hippocampus, these kinds of
%sensory or semantic overlap are discounted by orthogonalizing similar
%inputs encountered at different times.
Inspired by an understanding of neural representation, the project will
investigate how parallel lossy representations of the same inputs
along different dimensions can be used for statistically and computationally efficient learning algorithms, under the framework of exponential family embeddings and variational inference. The project will then investigate how machine learning models with multiple embeddings can be used for analysis of large and dynamic human brain imaging data.

\objective{Attentional filtering} The project will develop
attention-based models in statistical machine learning, based on the use
of lower-dimensional traces through a high-dimensional
input space. These traces have a close parallel in human cognition, where
sensory input dimensions are selected based on their inherent salience and
relevance to behavioral goals. The project will explore mathematical,
computational, and empirical models of attention. These models will then
be used to reconstruct the subjective experience of human subjects
watching naturalistic movie stimuli, based on their brain activity.

\objective{Memory capacity} Behavioral studies suggest that humans can store a massive number of 
objects and events in long-term memory with incredible detail.
To understand how this is possible in a finite brain, the project 
will use cognitive studies and memory architectures in natural systems 
to inform approaches for reducing and sharing memory in
artificial learning algorithms. A framework will be developed for
establishing lower bounds on the risk of machine learning algorithms under
memory constraints and for the use of memory landmarks in efficient nonconvex optimization. Insights from this mathematical theory will be
considered in the context of memory in complex organisms.

\vskip8pt

\noindent 
The {\it\bfseries intellectual merit} of the proposed research
derives from the extensive sharing of ideas and transfer of expertise between data science and neuroscience, with the goal of advancing 
knowledge and accelerating scientific progress in both domains. The {\it\bfseries broader impact} of the proposed research includes new cross-disciplinary collaborations within and between institutions, two open workshops aimed at an accessible level to build campus community, support and training of graduate students through student exchanges and co-mentorship, integration of undergraduate majors in data science and neuroscience including the development of new joint courses, new senior independent research options relevant for preparing the STEM workforce, opportunities for academic partnerships with the technology industry, and enhanced infrastructure for research by amplifying and bridging extant campus initiatives and resources including expanding the scope and impact of the original TRIPODS institute.

