
\msection{Results from TRIPODS Project}

\textbf{Jeff Brock} is currently supported as PI under NSF TRIPODS grant CCF-1740741, ''Foundations of Model Driven Discovery from Massive Data,'' from September 1, 2017 to August 31, 2020. The co-PIs of the grant are Stuart Geman, Eli Upfal, Bjorn Sandstede, and Joseph Hogan (Brown University). The total award amount was \$1,482,177.

The TRIPODS grant, in its first year, sought to bring researchers from diverse communities together for workshop activities, and likewise engaged remote researchers to come to campus for seminar talks and research engagement with affiliated faculty and students. We have recently hired 3 postdoctoral fellows to start on July 1, who will assist with achieving the goals of the project.

A fall 2017 workshop entitled Geometry and Topology of Data took place at ICERM in December 2017. This workshop was intended to bridge communities working in Topological Data Analysis, and methods of Diffusion Geometry to study high dimensional data sets. Impacts from workshop were extensive, leading to future collaborations that are the topic of three of our TRIPODS + X proposals, in neuroscience, in data science education, and in gene regulatory networks. We are tracking collaborations going forward, but list one paper of Oudot and Solomon, on which Oudot spoke at the conference (arXiv:1712.03630).

Brown's TRIPODS Institute has been focused on developing models for center related activities that build on intensive visits from outside scientists working in the focus areas of the Institute. In particular, the TRIPODS Institute sponsored short visits from researchers in the theme of topological data analysis to the TRIPODS sponsored Applied Topology Seminar. Highlights include visits from, Justin Curry, Attila Gyulassy, Lori Ziegelmeier, Carina Curto, Steve Oudot, Anthea Monod and Andrew Blumberg.

{\bf Training Activities.}
The visit from Oudot, of INRIA, served further the collaboration with Brock's graduate student Isaac Solomon. Oudot spoke on their work at the workshop, and Solomon is speaking at the Ohio State TRIPODS workshop on their joint work in late May 2018. According to Solomon, ''The general goal of our project is to use techniques from applied topology to analyze intrinsic metric structures, in particular metric graphs... Steve presented our ongoing work at the workshop, and I have presented on it at various conferences, including the upcoming TRIPODS workshop at Ohio next week.''

Sandstede's student Melissa McGuirl met with many of the TRIPODS visitors, who provided valuable research insights as well as career mentorship. In particular, she met with her co-Advisor, Blumberg, from U. Texas to discuss her thesis work. McGuirl also had substantial interactions with many of the speakers including Gyulassy of whom she says, ''We met to discuss ways to extract features from images. In particular we talked about Morse theory applications. We also discussed the current software tools available for this method.''

The PI has engaged in planning a week long summer bootcamp with Solomon, McGuirl, and Henry Adams of Colorado State, who will spend a week in early August presenting the framework of topological data analysis in the context of machine learning. This weeklong workshop (to run at ICERM) will focus on bringing graduate students up to speed with the tools and techniques of TDA, and bring in outside speakers to present their applications in to machine learning.

Finally, the TRIPODS Institute will sponsor a two day workshop in mid-August on building community in theoretical foundations of data science, focused on regional collaboration and interaction between researchers in the theoretical aspects of data science.
