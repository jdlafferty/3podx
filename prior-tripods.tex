
\msection{Results from TRIPODS Project}

\textbf{Jeffrey Brock} is currently supported as PI under NSF TRIPODS grant CCF-1740741, ''Foundations of Model Driven Discovery from Massive Data,'' from September 1, 2017 to August 31, 2020. The co-PIs of the grant are Stuart Geman, Eli Upfal, Bjorn Sandstede, and Joseph Hogan (Brown University). The total award amount was \$1,482,177. The TRIPODS grant, in its first year, sought to bring researchers from diverse communities together for workshop activities and intensive research visits. DSI and TRIPODS recently hired 3 shared postdoctoral fellows to start on July 1.

\textbf{Intellectual Merit} A fall 2017 workshop entitled Geometry and Topology of Data took place at ICERM in December 2017, bridging communities working in Topological Data Analysis, and methods of Diffusion Geometry. Connections made there have generated there TRIPODS + X proposals associated with Brown's TRIPODS grant, in neuroscience, in data science education, and in gene regulatory networks. 
The TRIPODS Institute sponsored short visits from researchers in the theme of topological data analysis to the TRIPODS sponsored Applied Topology Seminar. Highlights include visits from, J. Curry, A. Gyulassy, L. Ziegelmeier, C. Curto, S. Oudot, A. Monod and A. Blumberg.

The visit from Oudot, of INRIA, served to further the collaboration with PI Brock's graduate student Isaac Solomon. Oudot spoke on their work at the workshop, and Solomon spoke at the Ohio State TRIPODS workshop on their joint work  \citep{Oudot:Solomon:persistence} in late May 2018. The general goal of the project is to use techniques from applied topology to analyze intrinsic metric structures, in particular metric graphs. Sandstede's student Melissa McGuirl met with many of the TRIPODS visitors, including her co-Advisor, A. Blumberg, from U. Texas and A. Gyulassy, with whom she discussed extractio features from images using Morse theory. 

{\bf Education and Training Activities.}
Brock and Sandstede have engaged in planning a week long summer bootcamp with Solomon, McGuirl, and Henry Adams of Colorado State, who will spend a week in early August presenting the framework of topological data analysis in the context of machine learning. This week long workshop (to run at ICERM) will focus on bringing graduate students up to speed with the tools and techniques of TDA, and bring in outside speakers to present their applications in to machine learning.

{\bf Broader Impact.} The Geometry and Topology of Data workshop featured an open discussion over lunch to chart new directions for the field. The TRIPODS Institute will in addition sponsor a workshop in mid-August on building community in theoretical foundations of data science, focused on regional collaboration and interaction between researchers in the theoretical aspects of data science.
