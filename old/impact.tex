\mysection{Broader Impact}

Three aspects of the proposed research will have a broad impact,
extending beyond the intellectual contribution of the targeted
research.  First, as science becomes more data-driven, we see an
acute need for powerful nonparametric methods to handle
the increasing complexity of modern datasets.  Adequate
methodology is lacking, particularly methodology that
effectively incorporates realistic constraints from the problem at hand.
We expect to develop new families of flexible and principled
large scale data analysis tools that can benefit many
scientific domains.  Second, in our previous work
on some of the foundational problems of
machine learning and statistics, our research group and
collaborators have developed software that has been widely
distributed, allowing others to build on our work.  A 
recent example is the \texttt{huge} R package for high-dimensional
undirected graph estimation \citep{huge}. We will develop and 
make available software for all of the methods developed in
this project.  Third, we expect this research to have impact
across multiple communities, not only the statistics and
machine learning communities, 
but also in the domain sciences.


During the last three years we have developed
new courses at the University of Chicago that have reflected our
research in nonparametric statistics and large scale data analysis.
One class is ``Machine Learning and Large-Scale Data Analysis'' which
is taught at the advanced undergraduate and beginning graduate level.
Students from many different departments enroll.  This course includes
both standard lectures and exercises, but also a series of four large
scale data projects (LSD projects).  One of the LSD projects applies
semi-supervised learning to a dataset of 80 million images from
Google searches.  Another LSD project focuses on stochastic gradient
descent for fitting a logistic regression model on a streaming Twitter
feed.  Another involves kernel smoothing for a Poisson regression
model to predict crime rates from historical data in the city of
Chicago for one week during the course.  (These data are not sparse.)
The last concerns exoplanet finding from the Kepler telescope data,
which involves fitting nonparametric regressions, and running
hypothesis tests on the residuals, for the light curves of roughly
150,000 stars.  All of the computation is done in the cloud using a
Python interface to Amazon AWS that was developed specifically for
this course.  The costs are covered by an Amazon Machine Learning in
Education grant. A University of Chicago undergraduate developed this
infrastructure---he is now employed full time by Amazon AWS in
Seattle.  A second course is ``Nonparametric inference'' taught at the
advanced undergraduate and beginning graduate level.  Several
undergraduates have gotten interested in research in statistics and
machine learning through these courses, and have gone on to graduate
school in these areas.  Undergraduates have also initiated
research projects with the PI following these courses.  
We led an REU project during the summer of 2014 on
the theme of ``Learning Polynomials, Graphs and
Densities'' (\href{http://theorycenter.cs.uchicago.edu/REU/2014/projects.php}
{http://theorycenter.cs.uchicago.edu/REU/2014/projects.php}),
which has motivated part of Aim~3. Another such
project currently involves four undergraduates in a collaboration with
Zillow, the real estate company.  A new, advanced graduate course to
be offered in the Winter 2015 quarter is ``High dimensional
statistics.''  In other activity that broadens the impact of such research, the PI
recently co-organized a National Academy of Sciences workshop
called ``Training Students to Extract Value from Big Data''
(\href{http://www.nap.edu/catalog.php?record_id=18981}{http://www.nap.edu/catalog.php?record\_id=18981}).
Our work on high dimensional statistics informs all of these outreach
activities.

The broad range of topics in this research activity has the potential
to appeal to a broad range of students.  The PI currently supervises
eight Ph.D.~students, four of whom are women.  The University of Chicago
Department of Statistics is always looking for opportunities to
increase the diversity of the pool of students involved in statistical
research, including domestic students.  Our efforts
are aided by the 
University's commitment to increase the number of domestic students
from under-represented groups who enroll in and complete
Ph.D. programs---either at the University of Chicago or at other
national universities.  The University's \textit{Leadership Alliance}
program is one way of targeting this diversity.  Several other new and
existing activities in the division and at the University also
contribute to an overall goal to support diverse students in their
Ph.D. studies. These activities include the \textit{Master of Science
  in the Physical Sciences Division} (MS-PSD), the
\textit{Multicultural Graduate Community} (a new registered student
organization), a new \textit{UChicago chapter with SACNAS} (Society
for Advancement of Hispanics/Chicanos and Native Americans in
Science), \textit{CMEP Week} and \textit{Discover UChicago} (both
events for prospective graduate students).  Collectively, these
activities demonstrate that departmental efforts to recruit, prepare,
and retain diverse students in the statistical sciences will be
supported by wider divisional and University efforts.

%Recent NSF support has come from the following grants:
%III-Nonparametric Structure Learning for Complex Data Sets,
%July 2011--June 2014, Investigators: H. Liu, J. Lafferty, L. Wasserman;
%CCCF-0625879, MSPA-MCS: Nonparametric Learning in High Dimensions,
%August 2006--December 2010, Investigators: J. Lafferty, L. Wasserman
%and A. Lee;
%IIS-0427206, ITR: Collaborative Research (ACS+NHS)-(dmc+soc) Machine
% Learning for Sequences and Structured Data: Tools for Non-Experts,
%September 2004-2007, Investigators: John Lafferty, Andrew McCallum, Fernando
%Pereira; IIS-0312814,  ITR: Machine Learning from
%Labeled and Unlabeled Data,  September 2003--2006,
%Investigators: John Lafferty (PI) and Avrim Blum;
%IIS-0205456.
%%, NLP and Machine
%%Learning Techniques for Modeling Biological Sequences,
%%2001--2006, Investigators: John Lafferty (co-PI 
%%on subcontract to University of Pennsylvania),
%%Ken Dill, Aravind Joshi, Mark Liberman, Fernando Pereira;
%%CCR-0122581, ITR/AP: A Center
%%for ALgorithm ADaptation, Dissemination, and INtegration (ALADDIN),
%%September 2001--2006, Investigators 
%%Guy Blelloch (PI), Lenore Blum, John Lafferty,
%%Daniel Sleator, Ramamoorthi Ravi.


%Numerous publications arose from these grants; these are listed in a
%separate section of the references.  






