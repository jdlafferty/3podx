
\msection{Results from Prior NSF Support}

\textbf{Nick Turk-Browne} was previously supported as co-PI under NSF
grant BCS-1229597, ``MRI: Acquisition of High Performance Compute
Cluster for Multivariate Real-time and Whole-brain Correlation
Analysis of fMRI Data'' from August 15, 2012 to July 31, 2015. The PI
of the grant was Jonathan Cohen and the co-PIs were Ray Lee, Kai Li,
Kenneth Norman, and Turk-Browne (all Princeton University at the
time). The total award amount was \$527,978.

{\bf Intellectual Merit.} This project resulted in the acquisition of
a high-performance compute cluster for the analysis of human brain
imaging data, installed in the Princeton Neuroscience Institute (PNI).
The cluster consisted of 50 nodes, each with dual E5-2670 Xeon chips
(16 CPU cores) as host CPUs and 2 Xeon Phi 5110P boards (120 CPU
cores) as primary compute nodes, with 256GB DRAM, 1.6TB Flash memory,
30TB of local storage and 10GbE network interface (6,800 cores total).
The Xeon and Phi processors were donated by Intel Corporation as
in-kind cost sharing, valued as a similar amount to the award from
NSF. This system was integrated with state-of-the-art functional
magnetic resonance imaging (fMRI) equipment at PNI, enabling two
exciting long-term projects that required rapid-throughput analysis of
high-dimensionality data: closed-loop real-time neurofeedback and
whole-brain functional connectivity analysis. It resulted in several
publications, including
~\citep{Cohen:2017,deBettencourt:2015,Turk-Browne:2013,Wang:2015}.

{\bf Broader Impact.} This work brought together a diverse team of
neuroscientists, psychologists, physicists, and computer scientists,
enhancing not only the computing and imaging infrastructure at
Princeton, but also its intellectual and educational environment. Many
graduate students, postdoctoral fellows, and faculty have used this
cluster for neuroscience research. This includes the neuroscience and
computer science graduate students deBettencourt and Wang, who led
each of the projects mentioned above, respectively, resulting in
first-authored publications. These projects also resulted in the
development of BrainIAK, a high-performance open-source software
package for advanced fMRI analysis, as part of an industry partnership
with Intel Labs (brainiak.org).

\vskip10pt \noindent Turk-Browne was previously Co-PI of NSF grant
ACI-1440750, ``CC*IIE Engineer: A Software-Defined Campus Network for
Big-Data Sciences'' from September 1, 2014 to August 31, 2016. The PI
of the grant was Jennifer Rexford and the co-PIs were Curt Hillegas,
Christopher Tully, and Turk-Browne (all Princeton University at the
time). The total award amount was \$399,776.

{\bf Intellectual Merit.} This project resulted in the hiring of a
full-time cyber infrastructure engineer (CIE), who has been leading a
major campus initiative to deploy software-defined networking (SDN)
for the next generation of data-driven scientific research, including
in neuroscience, physics, geosciences, and genomics. This initiative
led to cost-effective scaling of the campus IPS, innovative solutions
to server load balancing, better monitoring of network performance
with perSONAR, and testing of OpenFlow switches.

{\bf Broader Impact.} One of the first tasks of the CIE was to create
a survey that interviewed faculty about their networking needs. This
survey and subsequent activities under this project led to a unique
collaboration between researchers in Computer Science and the
administrative Office of Information Technology, helping ensure that
academic research needs across campus were central to IT
infrastructure decisions.

%\vskip10pt \textbf{John Lafferty} was previously supported as co-PI under NSF
%grant IIS-1116730, ``III: Small: Nonparametric Structure Learning for
%Complex Scientific Datasets,'' from August 1, 2011 to July 31, 2014.
%The PI of the grant was Han Liu (Princeton University) and the co-PIs
%were Lafferty and Larry Wasserman (Carnegie Mellon University).  The
%total award amount was \$499,344; the amount awarded to the University
%of Chicago was \$118,750.
%
%{\bf Intellectual Merit.} %The research supported under this grant
%%contributed to the subfield of statistics called nonparametric
%%sparsity, which exploits novel sparsity-inducing regularizations to
%%fit nonparametric models. The project focused on developing scalable
%methods for finding structure in complex scientific datasets, without
%making strong distributional assumptions. The project explored several
%aspects of nonparametric structure learning, including methods,
%theory, large-scale computing, and applications, with five concrete
%aims: (1) nonparametric structure learning in high dimensions, (2)
%nonparametric conditional structure learning, (3) regularization
%parameter selection, (4) parallel and online nonparametric learning,
%and (5) minimax theory for nonparametric structure learning problems. 
%The outcomes included practical models and algorithms; application
%areas included genomics, cognitive neuroscience, climate science,
%astrophysics, and language processing.  Publications resulting from
%this grant include
%\citep{fcca,mehddtgm,chen:13,challenges,quadro,bigp,admm,NRRR:NIPS,gu:12,
%ospif,hdssipca,tpca,pcangd,rspcr,tmehdv,scale,pca,bernoulli,direct,
%coda,spcahdmts,peace,cna,biglasso,fshdc,kolarhan,gemad,gemd,
%lafferty2012,flare,hdtgm,liu2012,skeptic,expoconc,insensitive,mrc,
%eigens,calibrate,blossom,Mishra:2015,patterns,mngm,fclime,joint,retm,
%tests,shender:13,soft,latenttree,tar,sconvex,xu:14,csc,sicec,
%calibratedp,huge,hdngevspnp,semirank,amrbcdm,qnegsm,Zhu:18}.
%
%{\bf Broader Impact.} The broader impact of the project included
%interdisciplinary training for graduate students from biostatistics,
%computer science, statistics, and medical schools, strengthening the
%collaboration and interdisciplinary infrastructure between Carnegie
%Mellon, Johns Hopkins, the University of Chicago and Princeton, and
%broadly disseminating the results from this research in journals from
%all relevant fields. The research had impact outside of machine
%learning and statistics. In a genomic study, PI Liu applied structured
%nonparametric methods to analyze high dimensional genomic data,
%identifying several gene mutations associated with autism.  These
%results were published in Nature \citep{patterns}, and reported in the
%New York Times. In another neuroscience study, the PI developed an
%effective algorithm for predicting Attention Deficit Hyperactive
%Disorder (ADHD) disease.  The research led to several statistical
%software packages in R, including \citep{huge,fclime,flare}, all of
%which are freely available on CRAN.

\vskip10pt \textbf{John Lafferty} is currently supported as PI under NSF
grant DMS-1513594, ``Constrained Statistical Estimation and Inference:
Theory, Algorithms and Applications,'' from June 29, 2015 to July 1,
2018. The total award amount was \$320,000. After two years of the
project, the remainder of the funds were transferred from the
University of Chicago to Yale University, where the PI moved in July
2017.

{\bf Intellectual Merit.}  The project is studying constraints that
are present in complex scientific data analysis problems, but that
have not been thoroughly studied in traditional approaches. Different
aspects of theory, algorithms, and applications of statistical
procedures, with constraints imposed on the storage, runtime, shape,
energy or physics of the estimators and applications. The overall goal
of the research is to develop theory and tools that can help
scientists to conduct more effective data analysis. Publications under
this grant have included \citep{ChatterjeeL18,MishraILH18,
abs-1803-01302,MishraLH17,YangB0L16,ChatterjeeDLZ16,ZhengL16,
MishraZLH15,ZhengL15,ZhuL14,Bonak18}


{\bf Broader Impact.} The broader impact of the project is aimed in
three directions. First is the development of flexible and principled
large scale data analysis tools that can benefit many scientific
domains.  Second, is the development of software that is widely
distributed, allowing others to build on the work. The third is to
education, to allow the research to impact the training of students at
both the graduate and undergraduate levels.


\vskip10pt \noindent Lafferty was previously PI of NSF grant
DMS-1547396, ``RTG: Computational and Applied Mathematics in
Statistical Science'' from July 1, 2016 to July 1, 2017. This grant
did not transfer to Yale University; the current PI is Jonathan Weare
at the University of Chicago. The total award amount is \$1,697,320.

{\bf Intellectual Merit.}  This Research Training Group (RTG) project
supports creation of a dynamic, interactive, and vertically integrated
community of students and researchers working together in
computational and applied mathematics and statistics. The work is
motivated by the growing need to train the next generation of
statisticians and computational and applied mathematicians in new
ways, to confront data-centric problems in the natural and social
sciences.

{\bf Broader Impact.} The broader impact includes vertical integration
of education and training from undergraduate to postdoctoral
researchers, including activities at Toyota Technological Institute at
Chicago and Argonne National Laboratory. Participants in the RTG will
receive an educational experience that provides them with strong
preparation for positions in industry, government, and academics, with
an ability to adopt approaches to problem solving that are drawn from
across the computational, mathematical, and statistical sciences.

\vskip10pt \textbf{Damon Clark} is currently supported as PI under NSF
grant IOS-1558103, ''Understanding how Neural Nonlinearities Tune
Motion Detection in the Fly Eye,'' from July 1, 2016 to June 30, 2019.
The total award amount is \$461,262.

{\bf Intellectual Merit.} This project focuses on the nonlinearity
associated with visual motion detection circuits in fruit flies. Our
aims are to characterize that nonlinearity with behavioral and neural
measurements, dissect how individual neurons in the circuit contribute
to it, and investigate how that nonlinearity relates to the
performance of the circuit with naturalistic inputs. We have so far
published a review~\citep{clark:16} and a methods
paper~\citep{mano:17}, and currently have three more manuscripts
submitted or under review.

{\bf Broader Impact.} We have had 3 high school students work in lab
over the summers (2016-2018), and completed lessons in Yale's
\textit{Pathways to Science} program that focused on what visual
illusions teach us about how the brain works.


\textbf{Jeffrey Brock} (NSF DMS-1207572. Title: {\em Combinatorics, Models, and
  Bounds in Hyperbolic Geometry}. Amount: \$290,498, Period: 7/1/12 --
6/30/16.)  \textbf{Intellectual Merit.} 
Funded research on 
\cite{Brock:Canary:Minsky:elc,BBCM:pullout,
BBCL:dlt,Brock:Dunfield:inj,Brock:ams}.
 These papers build the bi-Lipschitz model manifold for Kleinian surface groups, and establish a number of related results including Gromov-Hausdorff convergence phenomena.
The paper \cite{Brock:Dunfield:inj} provides a counterexample to a Conjecture of Bergeron concerning the non-existence of integral homology spheres converging in the sense of
Benjamini-Schramm to hyperbolic 3-space.  
More rcently \cite{BMNS:bounded:models,Brock:Modami:nue,Brock:Bromberg:vol,Brock:Bromberg:cone:inflex}, give sufficient conditions for bounded geometry in terms of these bounded combinatorics, proving a hyperbolization and model theorem, and others give examples of Weil-Petersson geodesics in Teichm\"uller spaces of arbitrary genus that fail {\em Masur's criterion}. The recent paper
  \cite{Brock:Bromberg:vol} gives a new direct relation of Weil-Petersson distance and convex core volume.
%The submitted paper \cite{Brock:Dunfield:norms} compares the Thurston norm and the {\em harmonic norm} on the cohomology of   hyperbolic 3-manifolds showing they are roughly proportional and  the paper gives examples where the Thurston norm grows exponentially in  terms of the volume, while \cite{BLMR:teichmuller} analyzes new limiting phenomena in $\mathcal{PML}(S)$ for Teichm\"uller geodesics in the Thurston boundary. 
\textbf{Broader Impact.} During the funding period Brock was the managing Deputy Director in  charge of the organization of the ICERM Semester Program on {\em Topology, Geometry and Dynamics}. Brock also serves as a panelist in professional development roundtables at
  ICERM. Brock organized NSF funded conferences \textit{ Teichm\"uller Theory and Mapping Class Groups}, in Tel Aviv, Israel, as well as the NSF funded conference \textit{ Minimal Surfaces and Hyperbolic Geometry} which took place in Rio at IMPA in January, 2015.
%PI gave public lectures and mini-courses in topology, notably the \textit{AMS Current Events Bulletin } at the 2012 National Meetings in Boston.
PI served as the Postdoctoral Mentor to Sara Maloni from 2012-2016, and Tarik Aougab beginning 2015, as well as mentoring three women as graduate students. 