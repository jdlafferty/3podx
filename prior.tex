
\msection{Results from Prior NSF Support}

\textbf{John Lafferty} was previously supported as co-PI under NSF grant IIS-1116730,
``III: Small: Nonparametric Structure Learning for Complex Scientific
Datasets,'' from August 1, 2011 to July 31, 2014. The PI of the grant
was Han Liu (Princeton University) and the co-PIs were
Lafferty and Larry Wasserman (Carnegie Mellon University).  The
total award amount was \$499,344; the amount awarded to the University
of Chicago was \$118,750.

{\bf Intellectual Merit.}  
%The research supported under this grant
%contributed to the subfield of statistics called nonparametric
%sparsity, which exploits novel sparsity-inducing regularizations to
%fit nonparametric models. 
The project
focused on developing scalable methods for finding structure in
complex scientific datasets, without making strong distributional
assumptions. The project explored several aspects of nonparametric
structure learning, including methods, theory, large-scale computing,
and applications, with five concrete aims: (1) nonparametric structure
learning in high dimensions, (2) nonparametric conditional structure
learning, (3) regularization parameter selection, (4) parallel and
online nonparametric learning, and (5) minimax theory for
nonparametric structure learning problems.  The outcomes included
practical models and algorithms; application areas included genomics,
cognitive neuroscience, climate science, astrophysics, and language
processing.  Publications resulting from this grant
include 
\citep{fcca,mehddtgm,chen:13,challenges,quadro,bigp,admm,NRRR:NIPS,gu:12,ospif,hdssipca,tpca,pcangd,rspcr,tmehdv,scale,pca,bernoulli,direct,coda,spcahdmts,peace,cna,biglasso,fshdc,kolarhan,gemad,gemd,lafferty2012,flare,hdtgm,liu2012,skeptic,expoconc,insensitive,mrc,eigens,calibrate,blossom,Mishra:2015,patterns,mngm,fclime,joint,retm,tests,shender:13,soft,latenttree,tar,sconvex,xu:14,csc,sicec,calibratedp,huge,hdngevspnp,semirank,amrbcdm,qnegsm,Zhu:18}.


{\bf Broader Impact.}  
The broader impact of the project included interdisciplinary training
for graduate students from biostatistics, computer science, 
statistics, and medical schools, strengthening
the collaboration and interdisciplinary infrastructure between
Carnegie Mellon, Johns Hopkins, the University of Chicago and Princeton, and broadly
disseminating the results from this research in journals from all
relevant fields. The research had impact outside of machine learning and statistics.
In a genomic study, PI Liu applied structured nonparametric methods to analyze high
dimensional genomic data, identifying several gene mutations 
associated with autism.  These results were published
in Nature \citep{patterns}, and reported in the New York Times.
In another neuroscience study, the PI developed an effective
algorithm for predicting Attention Deficit Hyperactive Disorder
(ADHD) disease.  The research led to several 
statistical software packages in R, including
\citep{huge,fclime,flare}, all of which are freely available on CRAN.

\vskip10pt
\noindent
Lafferty is currently supported as PI under NSF grant DMS-1513594,
``Constrained Statistical Estimation and Inference: Theory, Algorithms
and Applications,'' from June 29, 2015 to July 1, 2018. The
total award amount was \$320,000. After two years of the project, the
remainder of the funds were transferred from the University of Chicago
to Yale University, where the PI moved in July 2017.

{\bf Intellectual Merit.}  The project is studying constraints that
are present in complex scientific data analysis problems, but that
have not been thoroughly studied in traditional approaches. Different
aspects of theory, algorithms, and applications of statistical
procedures, with constraints imposed on the storage, runtime, shape,
energy or physics of the estimators and applications. The overall goal
of the research is to develop theory and tools that can help
scientists to conduct more effective data analysis. Publications 
under this grant have included 
\citep{ChatterjeeL18,MishraILH18,
abs-1803-01302,MishraLH17,YangB0L16,ChatterjeeDLZ16,ZhengL16,MishraZLH15,ZhengL15,ZhuL14,Bonak18}


{\bf Broader Impact.}  
The broader impact of the project is aimed in three directions.
First is the development of flexible and principled
large scale data analysis tools that can benefit many
scientific domains.  Second, is the development of software
that is widely distributed, allowing others to build on the work.
The third is to education, to allow the research to impact
the training of students at both the graduate and undergraduate
levels.


\vskip10pt
\noindent
Lafferty was previously PI of NSF grant DMS-1547396,
``RTG: Computational and Applied Mathematics in Statistical Science''
from July 1, 2016 to July 1, 2017. This grant did not transfer
to Yale University; the current PI is Jonathan Weare at the 
University of Chicago. The total award amount is \$1,697,320.

{\bf Intellectual Merit.}  This Research Training Group (RTG) project
supports creation of a dynamic, interactive, and vertically integrated
community of students and researchers working together in
computational and applied mathematics and statistics. The work is
motivated by the growing need to train the next generation of
statisticians and computational and applied mathematicians in new
ways, to confront data-centric problems in the natural and social
sciences.

{\bf Broader Impact.}  
The broader impact includes vertical integration 
of education and training from undergraduate to postdoctoral
researchers, including activities at 
Toyota Technological Institute at Chicago and Argonne
National Laboratory. Participants in the RTG
will receive an educational experience that provides them with strong
preparation for positions in industry, government, and academics, with
an ability to adopt approaches to problem solving that are drawn from
across the computational, mathematical, and statistical sciences.

