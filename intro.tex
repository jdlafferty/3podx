\def\obj#1{\vskip5pt\noindent {\it\bfseries #1.}\enspace}

\msection{Introduction}

Neuroscience is a natural and productive application domain for data science:
the brain is a complex system whose measurement produces large amounts of
noisy, high-dimensional, and dynamic data; at the same time, the structure and
function of the brain, and the intelligent behavior it produces, have inspired
approaches in data science. This proposal brings together an interdisciplinary team,
spanning psychology, biology, mathematics, statistics, and computer science, to study
an interconnected set of objectives at the interface between neuroscience and data 
science. The underlying theme is to develop a two-way channel: In one direction, we 
will investigate how computational principles of data science can be used to understand 
recent empirical findings in neuroscience, at the cellular level in fruit flies and at 
the systems and cognitive level in humans. In the reverse direction, the project will 
view the processes and mechanisms of vision and cognition underlying these findings as a
source for new mathematical frameworks for data analysis.  

The research will focus on four interrelated objectives:
distributed processing, data representation, attentional filtering, 
and memory capacity. For each objective, a series of investigations
are proposed that bring together current research
frontiers in neuroscience, statistics, machine learning, and
computation.

\obj{Distributed processing}
The project will consider different models
for distributed processing, motivated by visual learning and perception 
in lower-level organisms. 

\obj{Data representation}
Inspired by current understanding of how information is stored in
different regions of the brain, the project will investigate how
parallel lossy representations of the same inputs along different
dimensions can be used for statistically and computationally efficient
learning algorithms.  In the other direction, we will investigate how
recent developments in machine learning might be used as mechanisms
for processing massive cellular and brain imaging data.

\obj{Attentional filtering} The project will explore mathematical,
computational, and biological models of attention. These will be developed 
as a framework for combating the ``curse of dimensionality'' in complex data. This 
will inspire analyses of datasets in which human subjects watched 
movies, using frame-by-frame annotations over movie dimensions to infer attentional 
focus and reconstruct subjective experience.

\obj{Memory capacity} 
We will consider cognitive studies and
current understanding of memory architectures in natural systems
to inform approaches for reducing and sharing memory in
artificial learning algorithms. A framework will be developed for
establishing lower bounds on the risk of machine learning algorithms under
memory constraints. A new framework for memory based on generative models
will be studied.
\vskip5pt

The project is motivated by the transfer of ideas between data science
and neuroscience, to advance knowledge in both domains. The
collaboration brings together perspectives and expertise from many
different disciplines, including five departments at Yale that have
not had substantive interactions: Psychology; Molecular, Cellular, and
Developmental Biology; Statistics and Data Science; Computer Science; and
Mathematics. Organized as a component of the NSF TRIPODS activity at
Brown University, it will establish a new bridge between Yale and
Brown, institutions that are a relatively short distance apart, and
that have made significant investments in neuroscience and data
science. Cross-disciplinary training of graduate students will be a
key component of the activity, and labs for a new undergraduate data
science course at Yale will introduce a neuroscience component that is
tied to the project.  Two workshops will be held during the course of
the project: one at Yale at the end of the first project year focused on neuroscience,
and the second at Brown during the second year focused on data science, hosted with 
TRIPODS and ICERM. Additional activities are detailed in the following sections.




