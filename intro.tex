\def\obj#1{\vskip5pt\noindent {\it\bfseries #1.}\enspace}

\msection{Introduction}

This proposal brings together an interdisplinary team, spanning
mathematics, statistics, computer science, psychology and biology,
to study an interconnected set of objectives at the 
interface between neuroscience and data science. The underlying theme is to develop
a two-way channel between data science and cellular and cognitive
neuroscience. In one direction, we will investigate how computational
principles of data science can be used to understand recent empirical
findings in neuroscience, associated with measurements at the cellular
level in fruit flies, and brain imaging studies in humans. In the
reverse direction, the project will view the processes and mechanisms
of vision and cognition underlying these findings as a source for
new mathematical frameworks for data analysis.  

The research will focus on four interrelated objectives:
distributed processing, data representation, attentional filtering, 
and memory capacity. For each objective, a series of investigations
are proposed that bring together current research
frontiers in neuroscience, statistics, machine learning, and
computation.


\obj{Distributed processing}
The project will consider different models
for distributed processing, motivated by learning and perception in
both lower-level organisms (visual processing in fruit flies) and
higher-level cognition (visual cognition in humans). 


\obj{Data representation}
Inspired by current understanding of representation of information in
different regions of the cortex, the project will investigate how
parallel lossy representations of the same inputs along different
dimensions can be used for statistically and computationally efficient
learning algorithms.  In the other direction, we will investigate how
recent developments in machine learning might be used as mechanisms
for processing massive cellular and brain imaging data.

\obj{Attentional filtering} The project will explore mathematical,
computational and empirical models of attention. One set of analyses will
focus on a public dataset of subjects watching episodes of Sherlock
while being scanned with fMRI, using a frame-by-frame annotation of
several dimensions of the movie as the basis for attention-based models.
Mathematical and statistical models of attention will be developed
as a framework for combatting the ``curse of dimensionality''
in complex data.

\obj{Memory capacity} 
We will consider cognitive studies and
a current understanding of possible memory architectures in natural systems
in order to inform approaches for reducing and sharing memory in
artificial learning algorithms. A framework will be developed for
establishing lower bounds on the risk of machine learning algorithms under
memory constraints. A new framework for memory based on generative models
will be studied.

The project is motivated by the transfer of ideas between data science and neuroscience,
to advance knowledge in both domains. It is organized as a component
of the NSF TRIPODS activity at Brown University. Cross-disciplinary
training of graduate students will be a key component of the
activity. In addition, labs for a new undergraduate data science
course at Yale will introduce a neuroscience component that is tied to
the project. Two workshops will be held during the course of the
project. At the end of 

